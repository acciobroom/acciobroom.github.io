%=======================02-713 LaTeX template, following the 15-210 template==================
%
% You don't need to use LaTeX or this template, but you must turn your homework in as
% a typeset PDF somehow.
%
% How to use:
%    1. Update your information in section "A" below
%    2. Write your answers in section "B" below. Precede answers for all 
%       parts of a question with the command "\question{n}{desc}" where n is
%       the question number and "desc" is a short, one-line description of 
%       the problem. There is no need to restate the problem.
%    3. If a question has multiple parts, precede the answer to part x with the
%       command "\part{x}".
%    4. If a problem asks you to design an algorithm, use the commands
%       \algorithm, \correctness, \runtime to precede your discussion of the 
%       description of the algorithm, its correctness, and its running time, respectively.
%    5. You can include graphics by using the command \includegraphics{FILENAME}
%
\documentclass[11pt]{article}
\usepackage{amsmath,amssymb,amsthm}
\usepackage{graphicx}
\usepackage[margin=1in]{geometry}
\usepackage{fancyhdr}
\usepackage{listings}
\usepackage{xcolor}

\setlength{\parindent}{0pt}
\setlength{\parskip}{5pt plus 1pt}
\setlength{\headheight}{13.6pt}
\newcommand\question[2]{\vspace{.25in}\hrule\textbf{#1: #2}\vspace{.5em}\hrule\vspace{.10in}}
\renewcommand\part[1]{\vspace{.10in}\textbf{(#1)}}
\newcommand\algorithm{\vspace{.10in}\textbf{Algorithm: }}
\newcommand\correctness{\vspace{.10in}\textbf{Correctness: }}
\newcommand\runtime{\vspace{.10in}\textbf{Running time: }}
\pagestyle{fancyplain}
\lhead{\textbf{\NAME}}
\chead{\textbf{Assignment \HWNUM}}
\rhead{CAP 5610  -- \today}

\makeatletter
\renewcommand*\env@matrix[1][*\c@MaxMatrixCols c]{%
	\hskip -\arraycolsep
	\let\@ifnextchar\new@ifnextchar
	\array{#1}}
\makeatother

\definecolor{codegreen}{rgb}{0,0.6,0}
\definecolor{codegray}{rgb}{0.5,0.5,0.5}
\definecolor{codepurple}{rgb}{0.58,0,0.82}
\definecolor{backcolour}{rgb}{0.95,0.95,0.92}

\lstdefinestyle{mystyle}{
	backgroundcolor=\color{backcolour},   
	commentstyle=\color{codegreen},
	keywordstyle=\color{magenta},
	numberstyle=\tiny\color{codegray},
	stringstyle=\color{codepurple},
	basicstyle=\ttfamily\footnotesize,
	breakatwhitespace=false,         
	breaklines=true,                 
	captionpos=b,                    
	keepspaces=true,                 
	numbers=left,                    
	numbersep=5pt,                  
	showspaces=false,                
	showstringspaces=false,
	showtabs=false,                  
	tabsize=2
}

\lstset{style=mystyle}




\begin{document}\raggedright
		\lstset{language=Python}  
	
	
%Section A==============Change the values below to match your information==================
\newcommand\NAME{Tsungai Chibanga}  % your name
\newcommand\HWNUM{1}              % the homework number


%%%%%%%%%%%%%%%%%%%%%%%%%%%%%%%%%%%%%%%%%%%%%%%%%%%%%%%%%%%%%%%%%%%%%%%%%%%%%%%%%%%%%%%%%%%%%%%%%%%%%%%%%%%%%%%%%%%%%%%%%%%

%%%%%%%%%%%%%%%%%%%%%%%%%%%%%%%%%%%%%%%%%%%%%%%%%%%%%%%%%%%%%%%%%%%%%%%%%%%%%%%%%%%%%%%%%%%%%%%%%%%%%%%%%%%%%%%%%%%%%%%%%%%

%%%%%%%%%%%%%%%%%%%%%%%%%%%%%%%%%%%%%%%%%%%%%%%%%%%%%%%%%%%%%%%%%%%%%%%%%%%%%%%%%%%%%%%%%%%%%%%%%%%%%%%%%%%%%%%%%%%%%%%%%%%
 
\question{1}{Classify the following attributes as binary, discrete, or continuous.  Further classify the attributes as nominal, ordinal, interval, ratio.} 


\part{a} \textbf{Rating of an Amazon product by a person on a scale of 1 to 5} 

Ordinal

\part{b} \textbf{The Internet Speed}

Ratio


\part{c} \textbf{Number of customers in a store.} 

Ratio

\part{d} \textbf{Your Student ID}

Nominal

\part{e} \textbf{Distance }

Ratio


\part{f} \textbf{Your letter grade (A, B, C, D)}

Ordinal

\part{g} \textbf{The temperature in the campus}

Interval




\question{2}{Distance/Similarity Measures} 
\part{a} \textbf{Which proximity measure would you use to group the boxes based on their shapes(length-width ratio)? Justify your answer.}


We would use correlation to group the boxes based on their shapes.

The $1 \times 2$ box is most similar to the $3 \times 6$ since the bigger box is 3 times the length-width of the smaller box.

The $1 \times 1$ box is most similar to the $3 \times 3$ since the bigger box is 3 times the length-width of the smaller box.


\part{b} \textbf{Which proximity measure would you use to group the boxes based on their size? Justify your answer.}

We would use the Euclidean distance to group the boxes by size.

The $1 \times 1$ has an area of 1 which is very close to the area of the $1 \times 2$ box which is 2.

The $3 \times 3$ has an area of 9 which can be grouped with the area of the $3 \times 6$ box which is 18.


\question{3}{Data Pre-processing of Titanic–Part 1} 

\part{a} \textbf{Which features are available in the dataset?}

PassengerId,   
Survived,         
Pclass,           
Name,           
Sex,             
Age,            
SibSp,         
Parch,          
Ticket,         
Fare,           
Cabin,           
Embarked        

\part{b} \textbf{Which features are categorical?}

Name,            
Sex,                            
Embarked        

\part{c} \textbf{Which features are numerical?}

PassengerId,      
Survived,         
Pclass,           
Age,            
SibSp,            
Parch,            
Fare           

\part{d} \textbf{Which features are mixed data types?}

Cabin, 
Ticket



\part{e} \textbf{Which features contain blank, null or empty values?}

In the training dataset: Age and Cabin

In the test dataset: Age and Cabin


\part{f} \textbf{What are the data types(e.g., integer, floats or strings) for various features?}

\begin{verbatim}
PassengerId      int64
Survived         int64
Pclass           int64
Name            object (string)
Sex             object (string)
Age            float64
SibSp            int64
Parch            int64
Ticket          object (mixed)
Fare           float64
Cabin           object (mixed)
Embarked        object (string)
\end{verbatim}


\part{g} \textbf{To understand what is the distribution of numerical feature values across the samples, please list the properties (count, mean, std, min, 25\% percentile, 50\% percentile, 75\% percentile, max) of numerical features}?

For the training set we have the following:

\begin{verbatim}
       PassengerId    Survived      Pclass         Age       SibSp  
count   891.000000  891.000000  891.000000  714.000000  891.000000   
mean    446.000000    0.383838    2.308642   29.699118    0.523008   
std     257.353842    0.486592    0.836071   14.526497    1.102743   
min       1.000000    0.000000    1.000000    0.420000    0.000000   
25%     223.500000    0.000000    2.000000   20.125000    0.000000   
50%     446.000000    0.000000    3.000000   28.000000    0.000000   
75%     668.500000    1.000000    3.000000   38.000000    1.000000   
max     891.000000    1.000000    3.000000   80.000000    8.000000   

            Parch        Fare  
count  891.000000  891.000000  
mean     0.381594   32.204208  
std      0.806057   49.693429  
min      0.000000    0.000000  
25%      0.000000    7.910400  
50%      0.000000   14.454200  
75%      0.000000   31.000000  
max      6.000000  512.329200  
\end{verbatim}

For the test set we have the following:

\begin{verbatim}
       PassengerId      Pclass         Age       SibSp       Parch        Fare
count   418.000000  418.000000  332.000000  418.000000  418.000000  417.000000
mean   1100.500000    2.265550   30.272590    0.447368    0.392344   35.627188
std     120.810458    0.841838   14.181209    0.896760    0.981429   55.907576
min     892.000000    1.000000    0.170000    0.000000    0.000000    0.000000
25%     996.250000    1.000000   21.000000    0.000000    0.000000    7.895800
50%    1100.500000    3.000000   27.000000    0.000000    0.000000   14.454200
75%    1204.750000    3.000000   39.000000    1.000000    0.000000   31.500000
max    1309.000000    3.000000   76.000000    8.000000    9.000000  512.329200
\end{verbatim}


\part{h} \textbf{To understand what is the distribution of categorical features, we define:  count is the total number of categorical values per column; unique is the total number of unique categorical values per column; top is the most frequent categorical value; freq is the total number of the most frequent categorical value. Pleaselistthe properties (count, unique, top, freq) of categorical features? }

For the training set we have the following:

\begin{verbatim}
male      577
female    314
Name: Sex, dtype: int64


S    644
C    168
Q     77
Name: Embarked, dtype: int64


1601                7
347082              7
CA. 2343            7
CA 2144             6
3101295             6
..
STON/O2. 3101271    1
8475                1
350035              1
349213              1
315097              1
Name: Ticket, Length: 681, dtype: int64


Vovk, Mr. Janko                            1
Lindblom, Miss. Augusta Charlotta          1
Sandstrom, Miss. Marguerite Rut            1
Stankovic, Mr. Ivan                        1
Ling, Mr. Lee                              1
..
Ohman, Miss. Velin                         1
Myhrman, Mr. Pehr Fabian Oliver Malkolm    1
Bracken, Mr. James H                       1
Beesley, Mr. Lawrence                      1
Herman, Miss. Alice                        1
Name: Name, Length: 891, dtype: int64


B96 B98        4
C23 C25 C27    4
G6             4
C22 C26        3
F2             3
..
B79            1
C118           1
C110           1
C62 C64        1
C91            1
Name: Cabin, Length: 147, dtype: int64

\end{verbatim}


For the testing dataset we have 

\begin{verbatim}
male      266
female    152
Name: Sex, dtype: int64


S    270
C    102
Q     46
Name: Embarked, dtype: int64


PC 17608      5
113503        4
CA. 2343      4
347077        3
C.A. 31029    3
..
17463         1
349235        1
32302         1
228414        1
315152        1
Name: Ticket, Length: 363, dtype: int64


Khalil, Mrs. Betros (Zahie Maria" Elias)"    1
Bird, Miss. Ellen                            1
Nilsson, Miss. Berta Olivia                  1
Keane, Mr. Daniel                            1
Shine, Miss. Ellen Natalia                   1
..
Khalil, Mr. Betros                           1
Bowen, Miss. Grace Scott                     1
Gilbert, Mr. William                         1
McGowan, Miss. Katherine                     1
Reynolds, Mr. Harold J                       1
Name: Name, Length: 418, dtype: int64


B57 B59 B63 B66    3
C116               2
C78                2
C6                 2
C101               2
..
C46                1
C32                1
D43                1
D21                1
D37                1
Name: Cabin, Length: 76, dtype: int64

\end{verbatim}

\clearpage

Code used for the homework



\begin{lstlisting}
import pandas as pd

pd.set_option('display.max_columns', 500)


train_df = pd.read_csv("./train.csv")
test_df = pd.read_csv("./test.csv")

combine = [train_df, test_df]

numerics = ['int16', 'int32', 'int64', 'float16', 'float32', 'float64']
numdf = train_df.select_dtypes(include=numerics)
nonNumdf = train_df.select_dtypes(exclude=numerics)

nonNumt = test_df.select_dtypes(exclude=numerics)


# nonNumdf =nonNumdf.drop(columns=['Name', 'Ticket'])

def nullFinder(df):
return df.isnull().sum()

def colTypes(df):
return df.dtypes

def sumStats(df):
return df.describe()

nullfind = map(nullFinder, combine)
columnTypes = map(colTypes, combine)
summaryStats = map(sumStats,combine)
print(list(nullfind))
print(*list(columnTypes), sep='\n')
print(*list(summaryStats), sep='\n')

nonNumdf['Sex'].value_counts()
nonNumdf.apply(pd.Series.value_counts)
\end{lstlisting}
\end{document}
